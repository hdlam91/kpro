\chapter{Database setup}\label{dbSetup}
The database dump we got from the customer was a 10.8GiB .bak file which contained the existing internal database of the adressa system - a MSSQL database.
We thus had to install MSSQL locally on our own computers to be able to restore this backup file before we could integrate the database into our project via the entity framework. To manage and restore the database, we had to use SQL Management Studio.\footnote{\href{http://www.microsoft.com/en-us/download/details.aspx?id=8961}{http://www.microsoft.com/en-us/download/details.aspx?id=8961}}

This package let us install the 2012 version of the management studio, or choose to update from an existing 2008 version of MSSQL Management Studio which it claimed was already installed. However, this package did only the install the SQL management studio but no SQL server instance was installed.

It was therefore impossible to connect to a MSSQL server instance because none existed, which made it impossible to restore the .bak file (database backup dump), because there were no SQL server instances to restore to.
We then tried to install "SQL server with tools express" which did in fact install a SQL server instance and we were allowed to connect to a it via the management studio, and we were able to click "restore database". We then navigated to the appropriate folder and chose the .bak file to restore. When we clicked "OK", it started to restore the database, but after a few minutes we got an error message saying that the database was too big to be restored, this was due to the express version can only restore a database up to 10.2GiB while the .bak file we got was 10.8GiB. Thus we had to install the non-express version to be able to restore the database. The last tool we tried to install was "Microsoft SQL Server 2012 Developer 32/64-bit", this installation had a server instance which we could restore the database to.\footnote{\href{http://www.katieandemil.com/sql-server-2012-restore-database-backup-file}{http://www.katieandemil.com/sql-server-2012-restore-database-backup-file}}
To integrate the database into our development tools (Visual studio using web api) we had to use the entity framework by adding an ADO.NET Entity data model\footnote{\href{http://www.entityframeworktutorial.net/EntityFramework5/entity-framework5-introduction.aspx}{http://www.entityframeworktutorial.net/EntityFramework5/entity-framework5-introduction.aspx}} of the database.
This let us treat database tables as objects, and it converts objects manipulation to SQL query syntax for the particular database.