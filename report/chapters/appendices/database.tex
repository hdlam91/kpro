1\section{Database setup}
The database dump we got from the customer was a 10.8GiB .bak file which contained the existing internal database of the adressa system, which is a MSSQL database.
We thus had to install MSSQL locally on our own computers to be able to restore this
 backup file before we could integrate the database into our project via the entity framework. To manage and restore the database, we had to use SQL Management Studio. Found here: \href{http://www.microsoft.com/en-us/download/details.aspx?id=8961}{http://www.microsoft.com/en-us/download/details.aspx?id=8961}

This package let us install the 2012 version of the management studio, or choose to update from an existing 2008 version of MSSQL Management Studio which it claimed was already installed. However, this package did not install an instance of the SQL server.

It was therefore impossible to connect to a MSSQL server instance because none existed, which made it impossible to restore the .bak file (database backup dump), because there were no SQL server instances to restore to.
We tried to install "SQL server with tools express" which did in fact install a SQL server instance and we were allowed to connect to it via the management studio, and we were able to click "restore database". We then navigated to the appropiate folder and chose the .bak file to restore. When we clicked "OK", it started to restore the database, but after a few minutes we got an error message saying that the database was too big to be restored, the express version can only restore a database up to 10.2GiB while the .bak file we got was 10.8GiB.
The last tool we installed was Microsoft SQL Server 2012 Developer 32/64-bit, it had a MSSQL server instance.
We could now restore the .bak file! http://www.katieandemil.com/sql-server-2012-restore-database-backup-file BAM!
By using the entity framework we could add a ADO.NET Entity data model of the database. (http://www.entityframeworktutorial.net/EntityFramework5/entity-framework5-introduction.aspx)