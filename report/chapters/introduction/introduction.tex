\section{Introduction}
\subsection{Project name - Automatic Import of Completed Ads}
The project is named \em Automated Importing of Completed Ads\em \ given by the \\
customer; \em Adressavisen AS.\em \ The team is to create a framework for automatic import of real estate ads
into Adresseavisens internal database and order system.
\subsection{Customer - Adresseavisen AS}
Adresseavisen is a regional newspaper in Trondheim, Norway. It publishes it's newspaper in Trøndelag and Nordmøre on a daily basis except for Sundays. It is an independent, conservative newspaper with a daily circulation of approximately 85000 NOK. 
Adresseavisen switched from broadsheet to tabloid format on 16 September 2006. Stocks in Adresseavisen are traded on the Oslo Stock Exchange.\\
"In addition to the main newspaper, Adresseavisen owns several smaller local newspapers in the Trøndelag region. They also own and operate a local radio station, Radio-Adressa, and a local TV station, TV-Adressa (prior to 30 January 2006: TVTrøndelag). They also have a stake in the national radio channel Kanal 24. In addition, the newspaper owns the local newspapers Fosna-Folket, Hitra-Frøya, Levanger-Avisa, Sør-Trøndelag, Trønderbladet and Verdalingen." \cite{adressaWiki}


\subsection{Project Purpose}
The purpose of this project is to create a framework for automated import of complete ads in the form of pdf-files. They will be accompanied by data, which shall be added to the database, and used to create an order in Adresseavisens internal order system for ads. This framework shall also be able to provide certain information that is to be given to customers when they wish to have an ad featured in Adresseavisen, such as available dates and ad-packages.


\subsection{Project background}
Adresseavisen currently uses an ad system where ads to be featured on website, or published in newspapers, are created in the system, with the use of one of very many templates. This is very complex , and an increasing amount of customers uses their own fully completed ads, in the form of pdf-files. 
Adresseavisen now has the need for a system which can automate and simplify the process of receiving these ads into the system.

\subsection{Problem Description}
Webassistenten will take a completed ad in the form of a pdf file and save it to the appropiate folder, these folder will have the correct ID of the customer so adressa can easily identify which pdf belongs to which customer.
Webassistenten will require certain data along with the pdf, ie. date, module etc. These data will be saved into a database internally in their system.

\subsection{Existing Technology}
Already existing technology that the customer could have used rather than employing us to develop new software.

\subsection{Stakeholders}
Customer:\\
Adresseavisen AS\\
\\
Customer representatives:\\
Asle Dragsten Liebech - \href{mailto://asle.dragsten.liebech@adresseavisen.no}{asle.dragsten.liebech@adresseavisen.no}\\
Jostein Danielsen - \href{mailto://josetein.danielsen@adresseavisen.no}{josetein.danielsen@adresseavisen.no}\\
\\
Group Members:\\
Audun Skjervold - <email>\\
Erlend Løkken Sigholt - <email>\\
Truls Hamborg - <email>\\
Hong-Dang Lam - <email>\\
\\
Group Supervisor:\\
Meng Zhu - \href{mailto://zhumeng@idi.ntnu.no}{zhumeng@idi.ntnu.no}
\\
\subsection{Measure of Project Effects}
We as a group will not see any difference if real estate companies uses this system, the end result of produced by or without our product will be the same. Hopefully the real estate companies will save time and the ad submitted is accepted in the format they want it to be.
\subsection{Duration}
The introduction to the course started on Wednesday 21.08.2013 when we all met in S6 for information and were introduced to the customer and the project.
The final delivery is due on 22.11.2013.