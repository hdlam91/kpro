\section{Conclusion and Future Work}

\subsection{Conclusion}

The project we were given did not contain any questions regarding thing to evaluate, so there are nothing to conclude in that respect. However, as a group we have come to several conclusions regarding the technology we used, and the project as a whole.
\\
While Web API has really practical automated functionality, when it works, it is a disadvantage that it requires tweaks and setup to work according to specific needs. It is also severly lacking in comprehensive and collected documentation and tutorials, and most of the available learning materials come in the form of videos and blog posts. This makes it a much harder technology to come to grips with than it initially seems. The use of Web API in the system might still be the most appropriate due to ease of integration into Adresseavisens systems. However: the group feels the need to make a note that other technologies might be more apppropriate, and easier to implement for the same type of service, if they provide better documentation. This especially applies in cases where the developers are not familiar with Web API, but might very well be more familiar with technological concepts used by other more widely used, or more general alternatives.

In terms of the project overall, we have come to some conclusions as well. The objective of modifiability and extensibility was not as easily achievable in some main parts of the project as initially thought, and as a consequence it was left more to aspects inherent in the technology (more on that in the Future Work section). The reason behind this is that the nature of such an application is rather intimately connected with its data types, class implementations, and the data communication interface (i.e. the database connection, which in our case was Entity Framework).

All in all, we conclude that the project has been moderately successful, with a potential for improvement in several aspects: The technology study, learning, and acclimatization parts probably should have been more emphasized, and included as a larger part of the project. The modifiability and extensibility part is possibly a whole project in itself to properly implement according to proper coding standards and practies. As a consequence, the part of the product that concerns that requirement could use some improvement. And finally, following the last point, the work required for modifiability and general implementations in Web API is more of a project in extending and covering up missing features in Web API than it is a subproject of the ad import system. 

\subsection{Future Work}

In terms of future work, we as a group have several items we could suggest for a future improvement of the product, after we have completed our part with this report.

We did not have the time to implement accepting and adding pictures in the database. This will require a module with some integration into our current project, but should not be
hard to accomplish.

Secondly we have mostly relied on the integrated extensibility in Web API to provide the desired modularity and extensibility in the product. Any future additions of modules could be
added via modification of WebAPI settings, and using our completed work as templates, but there is room for improvement in this regard. It's possible to generalize our API more, for support
of other ad types, though it would require significant work or much more experience with Web API than our group have had. The required work would possibly involve generalized Controller templates
who can handle various types of input and operations, and in that regard are extensible to usable Controllers for ads. This would also probably require generalized model binders and data formatters, which require more understanding of Web API than we have been able to get; it should nonetheless be achievable as a project in its own sense.

The project could also have benefited from a more purpose-built database, and by extension: a more straigthforward way to communicate with data storage. However, this would perhaps not be feasible to follow through on, due to the current database serving other functions and possibly serving other systems as well.
