
\section{Internal Risks} \label{internalRisks}
\subsection{Low experience with the development process}
\begin{table}[H]
\begin{tabular}{| l | l |}
	\hline
	What: & The group doesn't have much experience with lengthy projects\\
	\hline
	Probability: & High \\
	\hline
	Impact: & Low-High \\
	\hline
	Action: & Reduce by regularly reviewing progress and making use of \\
			& supervisor meetings etc.\\
	\hline
\end{tabular}
\caption{Risk: Low experience with the development process}
\end{table}

\subsection{Unfamiliarity with technology}
\begin{table}[H]
\begin{tabular}{| l | l |}
	\hline
	What: & Some members of the group have no experience with the \\
		& programming language in use, and only one has used the \\
		& relevant framework. Will require extra time for learning.\\
	\hline
	Probability: & Moderate \\
	\hline
	Impact: & Low-Moderate \\
	\hline
	Action: & Reduce by group members with experience \\
	& coaching others in the relevant technology\\
	& and practices. Set time aside for learning.\\
	\hline
\end{tabular}
\caption{Risk: Unfamiliarity with technology}
\end{table}

\subsection{Unfamiliarity with tools}
\begin{table}[H]
\begin{tabular}{| l | l |}
	\hline
	What: & Some members of the group have no experience with\\
	& the tools we're using, it's complicated to\\
	& install due to the different versions of the tools \\
	& \& there's so many tools needed to start.\\
	\hline
	Probability: & Moderate \\
	\hline
	Impact: & Low-Moderate \\
	\hline
	Action: & Reduce by group members with experience \\
	& coaching others in the relevant technology\\
	& and practices. Set time aside for learning.\\
	& Debugging via Google.\\
	\hline
\end{tabular}
\caption{Risk: Unfamiliarity with tools}
\end{table}

\subsection{Illness}
\begin{table}[H]
\begin{tabular}{| l | l |}
	\hline
	What: & As winter approaches, the probability of group members becoming \\	 & ill increases. Being a small group, this might be critical.\\
	\hline
	Probability: & Moderate \\
	\hline
	Impact: & Low-High \\
	\hline
	Action: & Do not tailgate deadlines. Work steadily, and have margins\\
	&and practices. Expect some members to not produce at 100\%\\	& every week.\\
	\hline
\end{tabular}
\caption{Risk: Illness}
\end{table}

\subsection{Other engagements}
\begin{table}[H]
\begin{tabular}{| l | l |}
	\hline
	What: & Group members have extracurricular activities that require time \\
	 & certain dates during the semester.\\
	 & This might cause absence and/or reduced work-output.\\
	\hline
	Probability: & Moderate \\
	\hline
	Impact: & Low \\
	\hline
	Action: & Plan for it well in advance. It seems however that the time required\\
	& is fairly concentrated and/or pre-planned, so it should be easy to\\
	& plan around.\\
	\hline
\end{tabular}
\caption{Risk: Other engagements}
\end{table}

\subsection{Other subjects}
\begin{table}[H]
\begin{tabular}{| l | l |}
	\hline
	What: & Customer driven projects isn't the only subject we're assigned to\\
	& this semester, the other subjects might have projects too so we might\\
	& not be able to allocate enough time for the project.\\
	\hline
	Probability: & High \\
	\hline
	Impact: & Moderate \\
	\hline
	Action: & Try to plan for it in advance. All the courses have\\
	 & clear deadlines and we should be able to plan ahead.\\
	\hline
\end{tabular}
\caption{Risk: Other subjects}
\end{table}

\subsection{Underestimation of implementation}
\begin{table}[H]
\begin{tabular}{| l | l |}
	\hline
	What: & We have estimated that the implementation scope is \\
	&not a significant majority of the project, and possibly \\
	&even smaller than the process and documentation parts. \\
	& If we somehow have misjudged this, we will face \\
	& significant delays/increased workload when the plan will have to\\
	& be adjusted.\\
	\hline
	Probability: & Low \\
	\hline
	Impact: & High \\
	\hline
	Action: & Ensure to plan for more time for the project than we expect to need.\\
	& Do not rush the pre-study, and familiarize ourselves sufficiently with\\
	& all aspects.\\
	& Ensure we have time to work overtime if necessary during the\\
	& implementation period/sprints.\\
	\hline
\end{tabular}
\caption{Risk: Underestimation of implementation}
\end{table}

\section{External Risks} \label{externalRisks}
\subsection{Deaths}
\begin{table}[H]
\begin{tabular}{| l | l |}
	\hline
	What: & There's always a chance of a family member passing away, leading to\\
	& absence.\\
	\hline
	Probability: & Low \\
	\hline
	Impact: & High \\
	\hline
	Action: & Be understanding, try to reschedule working hours.\\
	\hline
\end{tabular}
\caption{Risk: Deaths}
\end{table}

\subsection{Unreachable Customer}
\begin{table}[H]
\begin{tabular}{| l | l |}
	\hline
	What: & Customer might be on vacation, in a meeting or not able to respond\\
	& instantly.\\
	& Database setup\\ %TODO What?
	\hline
	Probability: & Medium \\
	\hline
	Impact: & Low \\
	\hline
	Action: & Wait and call again or send and email or short message service\\
	& Set up our own local database\\
	\hline
\end{tabular}
\caption{Risk: Unreachable Customer}
\end{table}
