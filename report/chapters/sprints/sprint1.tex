\section{Sprint 1}

\subsection{Time Frame}
The time frame for Sprint 1 was week 38 and 39. We started the sprint on September 16th with a weekly supervisor meeting, followed by a sprint plan meeting. We finished the sprint on September 27th, with the sprint review meeting postponed to the following Monday.

\subsection{Original Plan}
From our Work Breakdown Structure, we had the following tasks planned for this sprint:
\begin{itemize}
	\item Enable display of available products (no GUI is to be implemented)
	\item Listing the next 5 available booking dates for a selected product
	\item Listing modules available for a selected product
\end{itemize}

\subsection{Revised Plan}
At the sprint plan meeting, we added more tasks to our Trello board (see subsection \ref{Trello} in the chapter on Technology) beyond those from our WBS. Most notably, we added the following:
\begin{itemize}
	\item Database connection interface
	\item Database submission logic
\end{itemize}

We did not revise the plan for the first sprint beyond adding more tasks. Being quite unfamiliar with the process, we were unaware at the time that we would not be able to complete all - or even any - of the planned tasks.

\subsection{Development}
The development started off with the creation of an ASP.NET MVC project. In this project we created data objects for the ad info described in the assignment.

Towards the end of the sprint, we received a database dump from the customer, which we spent the entire rest of the sprint attempting to restore. Please refer to Appendix \ref{dbSetup} on page \pageref{dbSetup} for more information on this.

\subsection{Other Work}
In addition to the development, we also completed a draft of the outline for this report and improved on our architectural documentation by adding new and removing unused views.

\subsection{Backlog}
Of the tasks planned at the beginning of the sprint, the following remained in our backlog at the end of the sprint:
\begin{itemize}
	\item Enable display of available products (no GUI is to be implemented)
	\item Listing the next 5 available booking dates for a selected product
	\item Listing modules available for a selected product
	\item Database connection interface
	\item Database submission logic
\end{itemize}

These tasks were incidentally all of the tasks we had planned for the sprint, as we were a bit optimistic in our planning. Some of these were included in the revised plan for sprint 2, while others were postponed to be included in a later sprint.

\subsection{Customer Meeting}
This customer meeting which took place on Friday 20 September 2013 had 2 customer representatives - Asle Dragsteen and Hans Kristian Moen because both of them works with this technology at Adressa. 
We started the customer meeting by going through our preliminary study phase and all of its document - the architecture and user stories documentation. The customer was satisfied with the documentation we had created.\\
The customer proceeded by giving us the database dump and showed us the database structure and which tables we are supposed to work with.\\
We also discussed the which Clean Coding principle we should follow (refer to \ref{cleanCoding}Clean Coding principle on page \pageref{cleanCoding} for more info). The customer wanted "Uncle Bob Clean Code".

\subsection{Retrospective}
While we had initially planned to complete some of the "easy" parts of the implementation during this sprint, this turned out to be a lot harder than expected. We only managed to get started before we ran into problems related to unfamiliarity with the technology. We mostly used this sprint to set up various tools needed for the implementation.

We also spent less time both on development and on the project in general than we had originally planned, partly due to projects in other courses also taking up a lot of our time.

What we learned from this sprint was more about how much we would actually be able to do in a single sprint, and that we were still quite far away from being able to implement specific parts of the requirements. We started doubting that we would be able to finish in three sprints, but remained hopeful. We also learned a lot about the tools we used.
