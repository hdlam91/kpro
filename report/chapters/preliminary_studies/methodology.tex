\subsection{Development Methodology}
The course compendium proposes two types of development methodologies: the sequential method \emph{Waterfall}, and the agile method \emph{Scrum}. This subsection supplies a brief introduction to these two approaches, followed by our argumentation for and against the two approaches in the case of our particular project. This subsequently followed by a conclusion as to which approach(es) we chose for our project.
\subsubsection{Waterfall}
The Waterfall development method is a sequential design process. It is divided into clearly defined, mostly separated phases, although there often is some overlap between them. The first phases focus on gathering requirements and writing initial documentation like design/architecture. Later phases move on to actual implementation, then testing, followed by final report. Maintenance after a "completed" project might also in some cases be part of the process. <add figure depicting process>
\subsubsection{Scrum}
The scrum development method is an agile approach to development. It is an iterative process consisting of several cycles, each containing most of the phases of a sequential method, and resulting in a functioning prototype. <add figure depicting process>
\subsubsection{What we decided on}
Each of the proposed methods have their own advantages and disadvantages, that make each of the methods respectively a better or worse fit for our project than the other. Some of these advantages are applicable and beneficial to our project, while others are not, or are in such a small degree that they are negligible. Below we have highlighted the most important advantages and disadvantages of each method, leading to a conclusion and our choice of method. \\
\emph{Waterfall}
\begin{itemize}
\item Advantages
	\begin{itemize}
		\item It is suitable for small projects, since they are manageable to plan fully. This fits our project description.
		\item It's easier to get every involved party on the same page with a thorough plan. This is beneficial to any project, but even more so in a project like ours, where the team consists of students who may have other projects in other courses running in parallel with this.
	\end{itemize}
\item Disadvantages
	\begin{itemize}
		\item Sequential methods don't handle change to the requirements particularly well, making this approach risky in the case of the customer wishing to modify the requirements during the implementation phase.
	\end{itemize}
\item Advantageous, but not applicable to our project
	\begin{itemize}
		\item Waterfall is a good method if you know everything about the project beforehand, or are able to acquire the required information and the full project specification before the implementation phase. Due to the course schedule and the relatively limited time frame of our project, we felt we needed to start implementation earlier than a long planning and requirements phase would allow.
		\item Requires little underway feedback, which could fit if access to customer was restricted, but specifications are expected to remain the same. While the specifications for the project are expected to remain unchanged, our goal was to include the customer in the process.
		\item The method supplies strong documentation as the first phases are focused entirely on creating these documents. However, while the course relies heavily on documentation, the customer had no use for most of this documentation.
	\end{itemize}
\end{itemize}

\emph{Scrum}

\begin{itemize}
	\item Advantages
	\begin{itemize}
		\item Supports rapid production of prototypes to show the customer. This allows for easier correction of misunderstandings, because they become apparent earlier through the functioning prototypes.
		\item This type of approach is highly supported by online tools that let both parties (the team and the customer) stay up-to-date on the planning and prioritization of tasks during implementation.
	\end{itemize}
	\item Disadvantages
	\begin{itemize}
		\item Heavily reliant on easy access to the customer for continous feedback and extraction of requirements. This is okay for our project, as our customer is readily available through both e-mail and phone. %Disadvantage of the method, but not a problem for us?
		\item This approach was suggested by the customer, and is the same as they use. Using the same method as the customer might be beneficial to the project, as it improves communication and workflow.
		\item The method utilizes stand-up meetings, a scrum master, and optionally (and preferably) a kanban/scrum board. These things do carry overhead, but provide both the team and customer with frequent feedback, making it both a pro and a con.
		\item Agile methods handle change to requirements very well, due to the high underway involvement of the customer. The risk of weighting this point is the possibility of planning of unnecessary change, but as we expected there to be necessary changes underways, we decided this was a good fit for the project. % Que?
		\item Relies on experience with the full development process if it’s to cover the entirety of the project.
	\end{itemize}
	\item Advantageous, but not applicable to our project
	\begin{itemize}
		\item Dude
	\end{itemize}
\end{itemize}
From our analysis of the two methods, we concluded that neither were a perfect fit for every part of our project. We felt that our inexperience with projects such as this one made it too difficult to complete the entire project through an agile process, but we felt too unsure about the scope of the project to plan everything ahead and do a straight sequential process. We did, however, wish to plan an outline for the project, create a general architectural overview and gather the most important requirements before we started implementation.

This led us to a decision of employing the waterfall method, or at least something similar, for the complete process, with a relatively long period of planning before starting implementation. We also decided to complete the implementation phase as an agile process, divided into two-week sprints, each consisting of a sprint planning meeting, then several days of implementation, with semi-daily stand-up meetings, and ending in a sprint review meeting and a functioning prototype.