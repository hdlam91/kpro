\section{Technology}

\subsection{Windows 7, 8}
Microsoft Windows 7 and - 8 are operating systems by the Microsoft Corporation. They logically provide good support for .NET developments, seeing as .NET targets the Windows platform and is made by Microsoft. Visual Studio is made for Windows, and was our main IDE, so all team members had access to PCs with Windows installed.

\subsection{Ubuntu Linux}
This OS is perhaps the most widely used distribution of Linux, developed by Canonical Ltd. It provides good support for many development tools, except of course Windows development. However we did find support for using it for some Windows development.
This operating system was used by one team-member on a laptop, when working on-site at NTNU. For coding, Mono with MonoDevelop was used, while other tasks were mostly unaffected. The operating system provides good support for other parts of the process, such as Git and \LaTeX.

\subsection{Mac OSX}
%TODO

\subsection{Visual Studio 2012}
Visual Studio is Microsofts IDE for development for their platforms. This is the main IDE we developed the framework on, seeing as it has very good integration with C\# and .NET platforms, which we were required to use. We decided to use the ultimate version because this version provides everything we might need.

\subsection{.NET and Mono}
We were required to use ASP .NET MVC for our framework. ASP .NET MVC is a framework for web applications which enables the use of the Model View Controller (MVC) pattern. It is part of Microsofts .NET Framework suite, which is the preferred way of interactiong with Windows systems and OSes.

Mono is the open source-, cross-platform version of the .NET suite, which we used when not developing on Windows machines. It is available both for Windows, OS X, most Linux Distributions, Android, and various other operating systems.

\subsection{MonoDevelop}
This is an open source IDE for development with Mono, available for OS X and most Linux distributions. This was the IDE used when not developing on Windows machines.

\subsection{\LaTeX}
We quickly chose \LaTeX \ for our typesetting. It being the de-facto standard for academic typesetting, with good support for both code snippets, tables, references and bibliography.

Most of our group also had at least some experience using it, and some were quite experienced, which made the choice easier.

\subsection{Git}
For our version control and source repository, we chose Git. This because we had most experience with it, and found it easy to set up via GitHub (where we all had accounts already). It also has the advantage of being distributed, so we could avoid a single point of failure, and having a staging area where one can selectively commit files according to whether they're ready or not, instead of risking accidental changes which might break something.

Both the source code and the entirety of the report source files were stored on GitHub, since both would be catastrophical to lose, and were quite important to have under version control in case we needed to track problematic changes.

\subsection{Trello}
To support our agile process and sprints, we used Trello for planning and control of workflow. It is an online Kanban Board tool, where we can create work packages and issues, while tracking who does what, and tracking backlog, finished modules, and work in progress.

\subsection{Web API}
%TODO

\subsection{Dropbox}
\href{http://www.dropbox.com}{Dropbox} is a syncing service that let you choose a local folder on your machine that will be synced to the cloud. Dropbox lets you share folders and files inside a shared Dropbox folder.

We used Dropbox for sharing and synchronizing internal documents that usually were only useful for a limited time, but might be referenced later.

\subsection{Google Drive}
\href{https://drive.google.com/}{Google Drive} is Google's office pack. The difference between Drive and other office solutions (like Microsoft Office , OpenOffice/LibreOffice) is that Drive exist in the cloud and lets the user simultaneously work on a document.

Google Drive was used for simultaneous collaboration on documents, where the content was up for discussion, or it was advantageous to see what the others where writing.

\subsection{Microsoft SQL server}
The customer uses Microsoft SQL server as their database, and we got the database backup dump. %TODO

\subsection{Entity framework}
The customer prefers if we use the entity framework to connect to the database. The entity framework is an object-relational mapper that enables .net developer to access the database without writing the typical data-access code that developer typically need to write. EF lets the developer work with domain specific object and properties without worrying about the underlying database table.
